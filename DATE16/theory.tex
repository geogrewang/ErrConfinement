\section{Proposed Error Confinement Method} \label{sec:theory}
Assume that a set of data $d \in \setD = \{d_1, \ldots, d_K\}$ being produced by an application are distributed according to the probability mass function $P_d(d_k) = \mathrm{Pr}(d=d_k)$. 
Such data are being stored in a memory, which is affected by parametric variations causing errors (i.e. bit flips) 
in some of the bit-cells. Sure errors eventually result in erroneous data leading to a new data distribution $\bar{P}_{d_k}$. 
The impact of such faults can be quantified by using a relevant error cost metric which in 
many cases is the mean square error (MSE) defined as  

\begin{equation}
      \setC(\bar{d}) \triangleq \mathbb{E} \big\{ (d-\bar{d})^2 \big\}
\end{equation}

with the expectation taken over the memory input $d$. Our proposed method focuses on minimizing the MSE between the original stored data $d$ and the erroneous data $\bar{d}$ in case of a-priori information about the error $\setF$ through an error-mitigation function  $d^* = g(\setF)$ which can be obtained by solving the following optimization problem:

\begin{equation}
    \label{eq:corrFunc}
    d^* = g(\setF) \triangleq \argmin_{\bar{d}} \, \setC(\bar{d} \, | \, \setF).
\end{equation}

where, 
\begin{equation}
    \label{eq:corrFuncMMSE}
    \setC(\bar{d} \, | \, \setF) \triangleq \mathbb{E}\big\{ (d-\bar{d})^2 \, | \, \setF  \big\}
\end{equation}

Basic arithmetic manipulations show that the resulting correction function is given by 
 $g_\text{MMSE} = \mathbb{E}\{d[n] \, | \setF\}$. This essentially corresponds to the expected value of the original fault-free data. Such expected values can be eventually determined offline through Monte-Carlo simulations 
or analytically in case that the reference data distribution is known already as in many DSP applications.  
Note that the above function depends on the applied cost metric that is relevant 
for the target application and other functions may exist that can be found by following the above procedure. 
In our paper, we focus on MSE which is relevant for many applications and especially for our case study that we discuss later.
  
%The reference matrix, which contains the statistic values for the data processing, has the ability to correct wrong data once %upon error detection by \textit{replace} the wrong data with the referred one. To further ensure the output quality, the %reference matrix itself can be protected using advanced ECC scheme, which incurs low-overhead due to its small size. %In this %work, we do not consider the protection of reference matrix while also ignore the faults injected in ECC memories, so that a %straightforward comparison for two error correction schemes is performed. 
%The advantages of DA approach over ECC are:

%\begin{itemize}
% \item \textsl{Predicted accuracy}: Statistic data collected from specific application improves the accuracy of correction due %to a pre-knowledge of the application, which is not available in generic method such as ECC.
% \item \textsl{Multiple bits correction}: SECDED is only able to correct 1 bit for each protected data. In contrast, generalized %data gives a good approximation to the erroneous data, which shows its effects on multiple data bits.
% \item \textsl{Low cost}: Error correction using ECC involves large area and power overhead, which are not incurred by the %approximate data.
%\end{itemize}